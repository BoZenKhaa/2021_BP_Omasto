\chapter{Implementation}

This chapter presents the solution design, implementation, and shows its benchmark.
As a goal for this project, we set to implement Finite-Horizon Solver for MDPs only.

\section{Solution Design}
The package's design is created in line with proposed declarations present in the original repository's README.md and discussed with Zachary Sunberg.

The goal of the project is to create a POMDPs.jl-compatible interface for defining MDPs and POMDPs with finite horizons, and in particular:
\begin{itemize}
    \item Provide a way for value-iteration-based algorithms to start at the final-stage and work backward
    \item Be compatible with generic POMDPs.jl solvers and simulators
    \item Provide a Finite-Horizon wrapper for Infinite-Horizon MDPs.
    \item Be compatible with other interface extensions like constrained POMDPs and mixed observability problems.
\end{itemize}

\subsection{Interface}
Moreover, the interface should contain or require the implementation of the following methods:
\begin{itemize}
    \item \textit{HorizonLength(::Type{<:Union{MDP,POMDP}}) = InfiniteHorizon()}
    \begin{itemize}
        \item \textit{FiniteHorizon}
        \item \textit{InfiniteHorizon}
    \end{itemize}
    \item \textit{horizon(m::Union{MDP,POMDP})::Int}
    \item \textit{stage\_states(m::Union{MDP,POMDP}, t::Int)}
    \item \textit{stage\_stateindex(m::Union{MDP,POMDP}, t::Int, s)}
    \item \textit{stage\_actions(m::Union{MDP,POMDP}, t::Int, [s])}
\end{itemize}

\subsection{Utilities}
And also the following utility:

\begin{itemize}
    \item \textit{fixhorizon(m::Union{MDP,POMDP}, T::Int) creates one of}
    \begin{itemize}
        \item \textit{FiniteHorizonMDP{S, A} <: MDP{Tuple{S,Int}, A}}
        \item \textit{FiniteHorizonPOMDP{S, A, O} <: POMDP{Tuple{S,Int}, A, O}}
    \end{itemize}
\end{itemize}

\section{Implementation}

The implementation is at \cite{FHPOMDP} or APENDIX.

The Finite Horizon algorithm evaluates a given MDP problem using Value Iteration.  

\subsection{Solution approach}

 The Current solution consists of a solver \textit{mysolve(mdp)} iterating and evaluating epochs and \textit{FiniteHorizonPolicy} structure storing its results (value function matrix, matrix of Qs, policy, map of actions, MDP instance). 
 
%  \textbf{Finite-Horizon Policy}
% \begin{description}[1cm]
%     \item $Q$ 
%     \item $V$  
%     \item $policy$ 
%     \item $actions$ 
%     \item $include\:Q$ 
%     \item $MDP$ 
% \end{description}
 
 \textit{Mysolve} takes the Finite-Horizon MDP as input, initializes policy structure and value function matrix, and iterates from \textit{$(maximum\_horizon-1)^{th}$} to \textit{$1^{st}$} epoch. Each epoch is then evaluated using a value iteration algorithm, which is initialized with the following epoch's value function matrix. Results of each epoch are stored in \textit{FiniteHorizonPolicy}.
 
 \LinesNumbered
\begin{algorithm}
\SetAlgoLined
initialize $FiniteHorizonPolicy$ \\
initialize $V$ \\
\For{$epoch = horizon - 1;\ epoch > 0;\ epoch = epoch - 1$}{
    set \textit{epoch} global \\
    run 1 iteration of \textit{value\_iteration(MDP, epoch, V)} \\
    update \textit{FiniteHorizonPolicy} \\
}
return \textit{FiniteHorizonPolicy}
\caption{Finite-Horizon MDP Solver mysolve}
\end{algorithm}

In the current version of the algorithm, we have to set the epoch global to pass it to interface functions. The next planned version will no longer contain the global variable.

\subsection{Defining MDP}

To define the MDP instance, we have to define the methods declared in \textit{JuliaPOMDP/POMDPs.jl} \cite{JMLR:v18:16-300} or from another library's package. The Finite-Horizon MDP Solver further requires: \textit{stage\_states}, which returns states of a given epoch, \textit{stage\_actions}, which return actions of a given epoch, and \textit{stage\_stateindex}, which return indices of a given state for a given epoch. For example of properly defined MDP see directory \textit{/test/instances}.
 

\subsection{Future plan}

The following releases' goals are to improve the performance, implement Infinite-Horizon MDP wrapper, and implement Finite-Horizon solver for POMDPs.


\section{Benchmark}
To prove the efficiency of implemented Finite-Horizon solver, we present a benchmark of Finite-Horizon MDP instance solved by
\begin{itemize}
    \item Value Iteration implemented in JuliaPOMDP/DiscreteValueIteration.jl
    \item Finite-Horizon Value Iteration solver implemented in this project in JuliaPOMDP/FiniteHorizonPOMDPs.jl package.
\end{itemize}

\subsection{MDP Instances}

To compare the results obtained, the instance has \textit{num\_of\_states * finite\_horizon} states for both solvers.

The instances were defined as 1D GridWorld problem with goal states on its right and left ends. The possible actions are \textit{move to left} and \textit{move to right}. The cost of such a move is 1, and the reward for moving to the goal is -10. Goals are terminal states and can not be left. The discount factor is one and the noise 0.6.

The benchmark were be performed on 3 different instance sizes:
\begin{itemize}
    \item \textbf{Instance 1} $states = 999, horizon = 500$,
    \item \textbf{Instance 2} $states = 99, horizon = 50$, and
    \item \textbf{Instance 3} $states = 9, horizon = 5$.
\end{itemize}

\subsection{Comparison}

The benchmark resulted in following:

\subsubsection{Instance 1}

\begin{itemize}
    \item \textbf{Value Iteration} 183.729947 seconds (3.48 G allocations: 244.793 GiB),
    \item \textbf{Finite-Horizon Value Iteration}   0.566122 seconds (7.17 M allocations: 579.764 MiB)
\end{itemize}

\subsubsection{Instance 2}

\begin{itemize}
    \item \textbf{Value Iteration} 1.082094 seconds (4.41 M allocations: 292.613 MiB)
    \item \textbf{Finite-Horizon Value Iteration}   0.199167 seconds (241.17 k allocations: 14.231 MiB)
\end{itemize}

\subsubsection{Instance 3}

\begin{itemize}
    \item \textbf{Value Iteration} 0.643639 seconds (1.08 M allocations: 51.450 MiB)
    \item \textbf{Finite-Horizon Value Iteration}   0.200691 seconds (172.53 k allocations: 8.653 MiB)
\end{itemize}  

We tested each benchmark's results to be the same.

The benchmarks show that the acyclic graphs with the optimal backup policy result in optimal policy and perform precisely both time-wise and memory-wise. We expect the results to be even better in the following versions.