% arara: pdflatex: { synctex: yes }
% arara: makeindex: { style: ctuthesis }
% arara: bibtex

% The class takes all the key=value arguments that \ctusetup does,
% and a couple more: draft and oneside
\documentclass[twoside]{ctuthesis}

\DeclareUnicodeCharacter{2212}{-}
\usepackage[ruled,vlined]{algorithm2e}
\usepackage{float}

\documentclass{article}
\renewenvironment{description}[1][0pt]
  {\list{}{\labelwidth=0pt \leftmargin=#1
   \let\makelabel\descriptionlabel}}
  {\endlist}

\parindent=0pt

\ctusetup{
% 	preprint = \ctuverlog,
	mainlanguage = english,
	titlelanguage = english,
% 	mainlanguage = czech,
% 	otherlanguages = {czech,english},
	title-czech = {MDP algoritmy v POMDPs.jl},
	title-english = {MDP algorithms in POMDPs.jl},
	subtitle-czech = {Zejména Finite-Horizon MDP},
	subtitle-english = {Finite-Horizon MDPs in particular},
	doctype = B,
	faculty = F3,
	department-czech = {Katedra Počítačů},
	department-english = {Department of Computer Science},
	author = {Tomáš Omasta},
	supervisor = {Ing. Jan Mrkos},
	supervisor-address = {E-325, \\ Charles Square 13, \\ Prague 2},
% 	supervisor-specialist = {John Doe},
	fieldofstudy-english = {Open Informatics},
	subfieldofstudy-english = {Artificial Intelligence and Computer Science},
	fieldofstudy-czech = {Otevřená Informatika},
	subfieldofstudy-czech = {Základy umělé inteligence a počítačových věd},
	keywords-czech = {Finite-Horizon, MDP, Julia},
	keywords-english = {Finite-Horizon, MDP, Julia},
	day = 10,
	month = 1,
	year = 2021,
	specification-file = {ctutest-zadani.pdf},
% 	front-specification = true,
	front-list-of-figures = false,
	front-list-of-tables = false,
%	monochrome = true,
%	layout-short = true,
}

\ctuprocess

\addto\ctucaptionsczech{%
	\def\supervisorname{Vedoucí}%
	\def\subfieldofstudyname{Studijní program}%
}

\ctutemplateset{maketitle twocolumn default}{
	\begin{twocolumnfrontmatterpage}
		\ctutemplate{twocolumn.thanks}
		\ctutemplate{twocolumn.declaration}
		\ctutemplate{twocolumn.abstract.in.titlelanguage}
		\ctutemplate{twocolumn.abstract.in.secondlanguage}
% 		\ctutemplate{twocolumn.tableofcontents}
% 		\ctutemplate{twocolumn.listoffigures}
	\end{twocolumnfrontmatterpage}
}

% Theorem declarations, this is the reasonable default, anybody can do what they wish.
% If you prefer theorems in italics rather than slanted, use \theoremstyle{plainit}
\theoremstyle{plain}
\newtheorem{theorem}{Theorem}[chapter]
\newtheorem{corollary}[theorem]{Corollary}
\newtheorem{lemma}[theorem]{Lemma}
\newtheorem{proposition}[theorem]{Proposition}

\theoremstyle{definition}
\newtheorem{definition}[theorem]{Definition}
\newtheorem{example}[theorem]{Example}
\newtheorem{conjecture}[theorem]{Conjecture}

\theoremstyle{note}
\newtheorem*{remark*}{Remark}
\newtheorem{remark}[theorem]{Remark}

\setlength{\parskip}{5ex plus 0.2ex minus 0.2ex}

% Abstract in Czech
\begin{abstract-czech}
Do povědomí vědecké komunity se začíná prodírat nový programovací jazyk, Julia. Díky svému prudkému nárustu uživatelské báze a svému prostředí postavenému na open-source knihovnách, se dá očekávat, že se stave jedním z nejpopulárnějších jazyků. V této práci představíme Julii, MDP, teorii s nimi spjatou spolu s možnými přístupy k jejich řešení. Postupně ukážeme jejich výhody i nevýhody a nakonec se dostane k Finite-Horizon MDP, jako jednou z nejpokročilejších metod. Navrhneme design Finite-Horizon MDP rozhraní na základě koncepce z JuliaPOMDP/FiniteHorizonPOMDPs.jl \cite{FHPOMDP}, implementujeme jej a porovnáme s value iteration řešením DiscreteValueIteration.jl \cite{DVI} implementovaným v knihovně JuliaPOMDP's \cite{JuliaPOMDP}.
\end{abstract-czech}

% Abstract in English
\begin{abstract-english}
In the last couple of years, a new programming language, Julia, has emerged into the scientific community's consciousness. With the steep rise of its downloads and its environment built on open-source packages, it is expected to become one of the most popular languages. This project introduces Julia, MDPs, the theory behind it, and possible approaches to its solution. By reviewing its drawbacks and advantages, we gradually get to present Finite-Horizon MDPs as the advanced solution. We design Finite-Horizon MDP Interface according to declaration at JuliaPOMDP/FiniteHorizonPOMDPs.jl \cite{FHPOMDP}, implement the interface and benchmark it against JuliaPOMDP's \cite{JuliaPOMDP} value iterator solver DiscreteValueIteration.jl \cite{DVI}.

\end{abstract-english}

% Acknowledgements / Podekovani
\begin{thanks}
We thank our Supervisor Ing. Jan Mrkos for all the little things he has done for us in making this possible, particularly for his patience. Our thanks also belong to our family and our girlfriend for the support they provided us with, and finally, to FEE CTU for all joys and sorrows it enriched us with.
\end{thanks}

% Declaration / Prohlaseni
\begin{declaration}
I declare that this work is all my own work and I have cited all sources I have
used in the bibliography.

\medskip

Prague, \monthinlanguage{title} \ctufield{day}, \ctufield{year}

\vspace*{2cm}

Prohlašuji, že jsem předloženou práci vypracoval samostatně, a že jsem uvedl veškerou použitou literaturu.

\medskip

V Praze, \ctufield{day}.~\monthinlanguage{second}~\ctufield{year}
\end{declaration}

% Only for testing purposes
\listfiles
\usepackage[pagewise]{lineno}
\usepackage{lipsum,blindtext}
\usepackage{mathrsfs} % provides \mathscr used in the ridiculous examples

\begin{document}

\maketitle

%!TEX ROOT=ctutest.tex

\chapter{Introduction}

MDP's are one of the most well-known methods used for solving stochastic planning problems. In the last couple of years, a new programming language, Julia, has emerged into the scientific community's consciousness. While offering a high-level approach comparable to the one of Python, it also offers a similar speed of languages like C or C++ at the same time. As the new users come to this language, the demand for new tools rises. The subject of this work answers these calls and deals with implementing one of those methods, finite-horizon MDPs interface for Julia's POMDPs.jl package.

\section{Markov Decision Processes}
The Markov decision processes are a part of the optimization problem solvers.
The term MDP was probably first used in \cite{cite:1} and was derived from the name of Russian mathematician Andrey Markov who researched stochastic processes, on which are MDPs based \cite{wiki:1}.
The problems that are MDPs solving are described with the environment. The environment consists of the states and actions (with corresponding stochastic transitions) for which the agent either pays a cost or receives reward (negative cost). The agent's task is to minimize his reward. The problem can be finite or infinite according to the problem's description.

\section{Julia}
Julia is a young language that starts to receive traction for its features. According to https://star-history.t9t.io/#JuliaLang\julia the number of stars on its repository(https://github.com/JuliaLang/julia) tripled over the last 20 months and is rising steadily.
The development started in 2009 at MIT, and the project went public in 2012 [https://www.zdnet.com/article/is-julia-fastest-growing-new-programming-language-stats-chart-rapid-rise-in-2018/]. 
It combines the performance of low-level languages like Fortran or C with the high-level approach of Matlab, R and Python. In addition to that, Julia is dynamically typed, designed for parallelism and distributed computation, it offers multiple dispatch paradigms and is also reproducible while relying on community packages[https://julialang.org/].


\section{POMDPs.jl}
\textit{POMDPs.jl is a package providing a core interface for working with MDPs and POMDPs}\cite{JMLR:v18:16-300}x It offers the user a unified interface for custom problem definitions, problem interfaces for simpler ones, and multiple solvers for both MDPs and POMDPs. The package is managed by Zachary Sunberg and is under active development.


\section{Goal}
The goal of this project is to implement a finite horizon interface for MDPs in the POMDPs.jl package. To get to know how the package's environment works, develop possible test cases, solve them using a custom value iteration method, and then compare them with the already implemented method in POMDPs.jl. Devise a way to implement finite horizon for MDPs in the package and evaluate its correctness. Get feedback from the POMDPs.jl community, iterate the design, and submit a pull request to the corresponding repository.



%!TEX ROOT=ctutest.tex

\chapter{Theory}

In the previous chapter, we have briefly introduced MDPs. In this chapter, we will build on that intuition by reviewing a small part of the theory behind them, and we will also show possible approaches to their solving.

\section{Environment}
The environment that we introduced so far contains states, actions, transition probabilities, and costs for each action. However, we did not assign any rules to it! 
Without these rules, one can not create any connection between states and actions or actions and transition probabilities.

The environmental rules could range anywhere from the ones applied in the crossword puzzle game (fully observable, deterministic, sequential, static, discrete with single-agent) up to the ones applied in the taxi driving planning (partially observable, stochastic, sequential, dynamic, continuous with multiple agents). We will not describe the meaning of all specific rules in this paper, as it is not the project's task. In case the reader is interested, we refer to read  \cite{russel2010} PRIDAT STRANKY DO REFERENCE asi 45-48.

\newpage


For the sake of simplicity, let us assume the following properties of the given environment:
\begin{description}
  \item[$\bullet$ Fully observable] The agent has all information needed about the environment.
  \item[$\bullet$ Stochastic] The movement of the agent is not certain. The result of movement can be different from the desired one as a result of noise,
  \item[$\bullet$ Sequential] The current action is going to have an impact on future actions,
  \item[$\bullet$ Static] The state of the environment can not change,
  \item[$\bullet$ Discrete] The environment has a finite number of states and actions.
\end{description}



\section{MDP and its features}
With environmental properties sorted out, let us move on to MDPs themselves. \\
Following theory are taken over from \cite{Kolobov2012}.
We define MDPs, as in \cite{Kolobov2012}, slightly modified for our purpose.
  \\ \\

\begin{definition}[\textbf{Fully Observable Markov Decision Process (MDP)}]
A fully observable MDP is a tuple (S, A, D, T, R), where:
\begin{description}
  \item[$\bullet$ ] $S$ is the finite set of all possible states of the system, also called the state space;
  \item[$\bullet$ ] $A$ is the finite set of all actions an agent can take;
  \item[$\bullet$ ] $D$ is a finite or infinite sequence of the natural numbers of the form $(1, 2, 3, \ldots, T_{max})$ or $(1, 2, 3, \ldots)$ respectively, 
        denoting the decision epochs, also called time steps, at which agent needs to act;
  \item[$\bullet$ ] $T : S \times A \times S \rightarrow [ \,0, 1] \,$ is a transition function, a mapping specifying the probability $T(s_1, a, s_2)$ of going to state $s_2$ if action $a$ is executed when the agent is in state $s_1$ at time step $t$;
  \item[$\bullet$ ] $R : S \times A \times S \rightarrow R$ is a reward function that gives a finite numeric reward value $R(s_1, a, s_2)$ obtained when the system goes from state $s_1$ to state $s_2$ as a result of executing action $a$ at time step $t$.
\end{description}

\end{definition}

Every MDP is solving its problem according to its movement policies. As the number of policies itself can be high order polynomial, we are going to relax their definition as well.
\\ \\
\begin{definition}[\textbf{Markovian Policy}]
A probabilistic (deterministic) history-dependent policy $\pi: H \times A \rightarrow [ \,0, 1] \,(\pi: H \rightarrow A)$ is \textit{Markovian} if for any two histories $h_{s,t}$ and $h'_{s,t}$, both of which end in the same state $s$ at the same time step $t$, and for any action $a$, 
$\pi(h_{s,t}, a) = \pi(h'_{s,t}, a)$ $(\pi(h_{s,t}) = a$ if and only if $\pi(h_{s,t}) = a)$.
\end{definition}

Every state in every time step is then evaluated according to \textbf{value function}. A Markovian value function then corresponds to $ V: S \times D 
\rightarrow [ \,-\infty, \infty] \,$. In the following text we will refer to value function as $V(s, t)$ or $V(s)$.

The value function of a policy is then: \\ \\

\begin{definition}[\textbf{The Value Function of a Policy}]
Let $h_{s,t}$ be a history that terminates at state $s$ and time $t$. Let $R^{\pi_{h_{s,t}}}_{t'}$ be random variables for the amount of reward obtained in an MDP as a result of executing policy $\pi$ starting in state $s$ for all time steps $t'$ s.t. $t \leqslant t' \leqslant |D|$ if the MDP ended up in state $s$ at time $t$ via history $h_{s,t}$.The value function $V^{\pi}: H \rightarrow [ \,−\infty,\infty] \,$ \textit{of a history-dependent policy} $\pi$ is a utility function $u$ of the reward sequence $R^{\pi_{h_{s,t}}}_t, R^{\pi_{h_{s,t}}}_{t+1}, \ldots$ that one can accumulate by executing $\pi$ at time steps $t, t + 1,\ldots$  after history $h_{s,t}$. Mathematically, $V^{\pi} (h_{s,t}) = u(R^{\pi_{h_{s,t}}}_t , R^{\pi_{h_{s,t}}}_{t+1}, \ldots)$.
\end{definition}

Among the policies evaluated according to the previous rule, we will be able to find an \textbf{optimal MDP solution} denoted as $\pi^*$ with value $V^*$ called the optimal value function. Such optimal solution then satisfies $V^{*} (h) \geqslant V^{\pi} (h)$ for all histories $h \in H$.

One of the possible approaches to evaluate the value function is using Expected Linear Additive Utility.
\\
\begin{definition}[\textbf{Expected Linear Additive Utility}]
An \textit{expected linear additive utility} function is a function $u(R_t, R_{t+1}, \ldots) = E[ \,\sum_{t'=t}^{|D|} \gamma ^{t'−t} R_{t'}] \, = E[ \,\sum_{t'=0}^{|D|-t} \gamma^{t'} R_{t'+t}] \,$ that computes the utility of a
reward sequence as the expected sum of (possibly discounted) rewards in this sequence, where $gamma \geqslant 0$
is the discount factor.
\end{definition}

This approach eliminates the possibility of multiple different solution values resulting from multiple stochastic runs of the same policy. It employs the expected value and does not need to perform vectors equality comparison as its results are scalars.

It turns out that this approach is even better as it guarantees a fundamental property of MDPs called \textbf{The Optimality Principle}, according to which \textit{among the policies that are evaluated by the expected linear additive utility, there exists a policy that is optimal at every time step.}

\section{MDPs techniques}

This part will briefly introduce a few fundamental algorithms for MDPs solving: Brute-Force algorithm, Policy Iteration, Value Iteration, and show a possible solution for its disadvantages: Prioritizations. \\
After reading this section, the reader should know the possibilities of solving MDP and be ready for the following section, where we will discuss the finite horizon's advantages.

\subsubsection{Brute-Force Algorithm}
Brute-force algorithm, as its name suggests, is the one most naive algorithm, that is similar for all classes of problem solving. In the proccess of evaluating the problem, the method evaluates all possible combinations and chooses the best one.
And it is obviously not used in vast majority of cases for this reason as the Algorithm needs to evaluate $|S|^{A}$ policies.
$S$ in the previous equation represents states and $A$ actions.

However, as the number of states, or actions, rises, the number of such evaluation becomes huge. Another problem comes from the fact, that actions in the environment are not deterministic and often cyclic. Both of these problems results in steep complexity rise.

That is why another approach comes in.

\subsubsection{An iterative approach to policy evaluation}

The evaluation of cyclic environments requires suitable equation which captures all movements, transition and rewards in it.
That means: value function of every state should correspond to sum of rewards for moving towards successor states. Moreover, the agent should maximize its reward by getting to goal, so the value function of the successor state multiplied by its probability should also appear in equation. This formula can be written as:\\
\begin{equation}
\begin{aligned}
V^{\pi} (s) & = 0 && \text{(if $s \in G$)} \\
& = sum_{s' \in S} T(s, \pi (s), s') [ \,C(s, \pi(s), s') + V^{\pi} (s')] \, && \text{(otherwise)}
\end{aligned}
\end{equation}

This system of linear equation reduced the complexity of previous approach. It \textit{can} be solved using Gaussian Elimination in $O(|S|^3)$. It is still not enough efficient, but its idea can be used to solve it with \textbf{an iterative approach}.

The iterative approach does exactly what it says. 
It starts with rough estimations and continuously works its way up to the same solution as of the linear system. In addition, this solution is also optimal \cite{Kolobov2012}. As It stores previous iteration it can be described as a part of dynamic programming. Every iteration of this algorithm runs in $O(|S|^2)$.

\LinesNumbered
\begin{algorithm}[H]
\SetAlgoLined
//Assumption: $\pi$ is proper (ends up in a goal state) \\
initialize $V^{\pi}_0$ arbitrarily for each state \\
$n \xleftarrow{} 0$ \\
\Repeat{$ max_{s\in S} residual_n (s) \leq \epsilon$}{
$n \xleftarrow{} n + 1$ \\
\ForEach{$ s \in S $}{
compute $V^{\pi}_n (s) \xleftarrow{} \sum_{s' \in S} T (s, \pi (s), s') [ \,C(s, \pi (s), s') + V^{\pi}_{n−1} (s') ] \,$ \\
compute $residual_n (s) \xleftarrow{} |V^{\pi}_n (s) − V^{\pi}_{n−1} (s)| $ \\
}
}
return $V^{\pi}_n$
\caption{Iterative Policy Evaluation}
\end{algorithm}

Now we know how to evaluate the states of the environment given some policy, but we did not yet described how to choose the best policy for given evaluation. 

\subsubsection{Policy iteration}
Given the fact that result of iterative policy evaluation is optimal after undefined number of iterations for specific policy, we can further improve it by iterating between policy evaluation and policy improvement.
For every iteration, we have \textif{almost} optimal policy evaluation for previous policy and with that we can improve the policy by iterating over possible actions from all states and choose that action, if is has better results, than previous one. The algorithm stops when the policy remains without a change at the end of following iteration.

\LinesNumbered
\begin{algorithm}[H]
\SetAlgoLined
initialize $\pi_0$ to be an arbitrary proper policy \\
$n \xleftarrow{} 0$ \\
\Repeat{$\pi_n == \pi_{n−1}$ }{
    $n \xleftarrow{} n + 1$ \\
    Policy Evaluation: compute $V^{\pi_{n−1}}$ \\
    Policy Improvement: \\
    \ForEach{state $s \in S$}{
        $ \pi_n (s) \xleftarrow{} \pi_{n−1} (s) $ \\
        $ \forall a \in A$ compute $Q^{(V^{\pi_{n−1}})} (s, a) $ \\
        $ V_n (s) \xleftarrow{} min_{a \in A} Q^{(V^{\pi_{n−1}})} (s, a) $ \\
        \uIf{$ Q^{(V^{\pi_{n−1}})} (s, \pi_{n−1} (s)) > V_n (s)$}{
            $ \pi_n (s) \xleftarrow{} argmin_{a \in A} Q^{(V^{\pi_{n−1}})} (s, a)$
        }
    \textbf{end}
    }
}
return $\pi_n$
\caption{Policy Iteration}
\end{algorithm}


\\
\\
\\
\\
\\
\\
\\
\\
\\
\\


\LinesNumbered
\begin{algorithm}[H]
\SetAlgoLined
initialize $V_0$ arbitrarily for each state \\
$n \xleftarrow{} 0$ \\ 
\Repeat{$max_{s \in S} residual_n (s) \let \epsilon$}{$n \xleftarrow{} n + 1$ \\\ForEach{$s \in S$}{compute $V_n (s)$ using Bellman backup at $s$\\ compute $residual_n (s) = |V_n (s) - V_{n-1} (s)|$}}
return greedy policy: $\pi^{V_n} (s) = argmin_{a\inA} \sum_{s'\inS} \tau(s, a, s')[ \,C(s, a, s') + V_n (s')] \,$
\caption{Value Iteration}
\end{algorithm}


\LinesNumbered
\begin{algorithm}[H]
\SetAlgoLined
initialize $V_0$ arbitrarily for each state \\
$n \xleftarrow{} 0$ \\
\Repeat{$ max_{s \in S} residual_n (s) \let \epsilon $}{}
return greedy policy: $\pi^{V_n} (s) = argmin_{a\inA} \sum_{s'\inS} \tau(s, a, s')[ \,C(s, a, s') + V_n (s')] \,$
\caption{Value Iteration}
\end{algorithm}





\chapter{Implementation}

This chapter presents the solution design, implementation, and shows its benchmark.
As a goal for this project, we set to implement Finite-Horizon Solver for MDPs only.

\section{Solution Design}
The package's design is created in line with proposed declarations present in the original repository's README.md and discussed with Zachary Sunberg.

The goal of the project is to create a POMDPs.jl-compatible interface for defining MDPs and POMDPs with finite horizons, and in particular:
\begin{itemize}
    \item Provide a way for value-iteration-based algorithms to start at the final-stage and work backward
    \item Be compatible with generic POMDPs.jl solvers and simulators
    \item Provide a Finite-Horizon wrapper for Infinite-Horizon MDPs.
    \item Be compatible with other interface extensions like constrained POMDPs and mixed observability problems.
\end{itemize}

\subsection{Interface}
Moreover, the interface should contain or require the implementation of the following methods:
\begin{itemize}
    \item \textit{HorizonLength(::Type{<:Union{MDP,POMDP}}) = InfiniteHorizon()}
    \begin{itemize}
        \item \textit{FiniteHorizon}
        \item \textit{InfiniteHorizon}
    \end{itemize}
    \item \textit{horizon(m::Union{MDP,POMDP})::Int}
    \item \textit{stage\_states(m::Union{MDP,POMDP}, t::Int)}
    \item \textit{stage\_stateindex(m::Union{MDP,POMDP}, t::Int, s)}
    \item \textit{stage\_actions(m::Union{MDP,POMDP}, t::Int, [s])}
\end{itemize}

\subsection{Utilities}
And also the following utility:

\begin{itemize}
    \item \textit{fixhorizon(m::Union{MDP,POMDP}, T::Int) creates one of}
    \begin{itemize}
        \item \textit{FiniteHorizonMDP{S, A} <: MDP{Tuple{S,Int}, A}}
        \item \textit{FiniteHorizonPOMDP{S, A, O} <: POMDP{Tuple{S,Int}, A, O}}
    \end{itemize}
\end{itemize}

\section{Implementation}

The implementation is at \cite{FHPOMDP} or APENDIX.

The Finite Horizon algorithm evaluates a given MDP problem using Value Iteration.  

\subsection{Solution approach}

 The Current solution consists of a solver \textit{mysolve(mdp)} iterating and evaluating epochs and \textit{FiniteHorizonPolicy} structure storing its results (value function matrix, matrix of Qs, policy, map of actions, MDP instance). 
 
%  \textbf{Finite-Horizon Policy}
% \begin{description}[1cm]
%     \item $Q$ 
%     \item $V$  
%     \item $policy$ 
%     \item $actions$ 
%     \item $include\:Q$ 
%     \item $MDP$ 
% \end{description}
 
 \textit{Mysolve} takes the Finite-Horizon MDP as input, initializes policy structure and value function matrix, and iterates from \textit{$(maximum\_horizon-1)^{th}$} to \textit{$1^{st}$} epoch. Each epoch is then evaluated using a value iteration algorithm, which is initialized with the following epoch's value function matrix. Results of each epoch are stored in \textit{FiniteHorizonPolicy}.
 
 \LinesNumbered
\begin{algorithm}
\SetAlgoLined
initialize $FiniteHorizonPolicy$ \\
initialize $V$ \\
\For{$epoch = horizon - 1;\ epoch > 0;\ epoch = epoch - 1$}{
    set \textit{epoch} global \\
    run 1 iteration of \textit{value\_iteration(MDP, epoch, V)} \\
    update \textit{FiniteHorizonPolicy} \\
}
return \textit{FiniteHorizonPolicy}
\caption{Finite-Horizon MDP Solver mysolve}
\end{algorithm}

In the current version of the algorithm, we have to set the epoch global to pass it to interface functions. The next planned version will no longer contain the global variable.

\subsection{Defining MDP}

To define the MDP instance, we have to define the methods declared in \textit{JuliaPOMDP/POMDPs.jl} \cite{JMLR:v18:16-300} or from another library's package. The Finite-Horizon MDP Solver further requires: \textit{stage\_states}, which returns states of a given epoch, \textit{stage\_actions}, which return actions of a given epoch, and \textit{stage\_stateindex}, which return indices of a given state for a given epoch. For example of properly defined MDP see directory \textit{/test/instances}.
 

\subsection{Future plan}

The following releases' goals are to improve the performance, implement Infinite-Horizon MDP wrapper, and implement Finite-Horizon solver for POMDPs.


\section{Benchmark}
To prove the efficiency of implemented Finite-Horizon solver, we present a benchmark of Finite-Horizon MDP instance solved by
\begin{itemize}
    \item Value Iteration implemented in JuliaPOMDP/DiscreteValueIteration.jl
    \item Finite-Horizon Value Iteration solver implemented in this project in JuliaPOMDP/FiniteHorizonPOMDPs.jl package.
\end{itemize}

\subsection{MDP Instances}

To compare the results obtained, the instance has \textit{num\_of\_states * finite\_horizon} states for both solvers.

The instances were defined as 1D GridWorld problem with goal states on its right and left ends. The possible actions are \textit{move to left} and \textit{move to right}. The cost of such a move is 1, and the reward for moving to the goal is -10. Goals are terminal states and can not be left. The discount factor is one and the noise 0.6.

The benchmark were be performed on 3 different instance sizes:
\begin{itemize}
    \item \textbf{Instance 1} $states = 999, horizon = 500$,
    \item \textbf{Instance 2} $states = 99, horizon = 50$, and
    \item \textbf{Instance 3} $states = 9, horizon = 5$.
\end{itemize}

\subsection{Comparison}

The benchmark resulted in following:

\subsubsection{Instance 1}

\begin{itemize}
    \item \textbf{Value Iteration} 183.729947 seconds (3.48 G allocations: 244.793 GiB),
    \item \textbf{Finite-Horizon Value Iteration}   0.566122 seconds (7.17 M allocations: 579.764 MiB)
\end{itemize}

\subsubsection{Instance 2}

\begin{itemize}
    \item \textbf{Value Iteration} 1.082094 seconds (4.41 M allocations: 292.613 MiB)
    \item \textbf{Finite-Horizon Value Iteration}   0.199167 seconds (241.17 k allocations: 14.231 MiB)
\end{itemize}

\subsubsection{Instance 3}

\begin{itemize}
    \item \textbf{Value Iteration} 0.643639 seconds (1.08 M allocations: 51.450 MiB)
    \item \textbf{Finite-Horizon Value Iteration}   0.200691 seconds (172.53 k allocations: 8.653 MiB)
\end{itemize}  

We tested each benchmark's results to be the same.

The benchmarks show that the acyclic graphs with the optimal backup policy result in optimal policy and perform precisely both time-wise and memory-wise. We expect the results to be even better in the following versions.


\chapter{Conclusion}

This thesis deals with the contribution to open-source Julia package \newline \textit{JuliaPOMDP/FiniteHorizonPOMDPs.jl} \cite{FHPOMDP} as a part of the more splendid library \textit{JuliaPOMDP} \cite{JuliaPOMDP}. Its task is to implement an interface for Finite-Horizon evaluations of MDPs. We introduced MDPs and their features, then built the essential MDP solvers theory, showed their advantages and disadvantages, and further improved these methods with prioritization. On its intuition, we finally presented the Finite-Horizon MDPs and summarized why it could be beneficial to use them instead of the Infinite-Horizon ones. 

We also benchmarked our package against the value iteration solver and showed better time and memory performance.


We believe that the package will be of fair use to everyone out there searching for a way to solve their Finite-Horizon MDP  problem with an already written package that is also efficient. The \textit{JuliaPOMDP} \cite{JuliaPOMDP} is an excellent designed general package for work with MDPs, and with the advent of Julia, it will get even more attention.

In future work, our main goal is to add POMDPs solver, improve the algorithm's efficiency and to create a package rewarding the user for its use.

In conclusion, our project is on its way to deliver a great package and further improve its library environment. With the rise of Julia, it is the best time to contribute to Julian open-source packages.

\appendix

\printindex

\appendix

\bibliographystyle{amsalpha}
\bibliography{ctutest}

% \ctutemplate{specification.as.chapter}

\end{document}