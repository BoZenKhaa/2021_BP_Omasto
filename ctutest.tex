% arara: pdflatex: { synctex: yes }
% arara: makeindex: { style: ctuthesis }
% arara: bibtex

% The class takes all the key=value arguments that \ctusetup does,
% and a couple more: draft and oneside
\documentclass[twoside]{ctuthesis}

\DeclareUnicodeCharacter{2212}{-}
\usepackage[ruled,vlined]{algorithm2e}
\usepackage{float}

%\documentclass{article}
\renewenvironment{description}[1][0pt]
  {\list{}{\labelwidth=0pt \leftmargin=#1
   \let\makelabel\descriptionlabel}}
  {\endlist}

\parindent=0pt

\ctusetup{
% 	preprint = \ctuverlog,
	mainlanguage = english,
	titlelanguage = english,
% 	mainlanguage = czech,
% 	otherlanguages = {czech,english},
	title-czech = {MDP algoritmy v POMDPs.jl},
	title-english = {MDP algorithms in POMDPs.jl},
	subtitle-czech = {Zejména Finite-Horizon MDP},
	subtitle-english = {Finite-Horizon MDPs in particular},
	doctype = B,
	faculty = F3,
	department-czech = {Katedra Počítačů},
	department-english = {Department of Computer Science},
	author = {Tomáš Omasta},
	supervisor = {Ing. Jan Mrkos},
	supervisor-address = {E-325, \\ Charles Square 13, \\ Prague 2},
% 	supervisor-specialist = {John Doe},
	fieldofstudy-english = {Open Informatics},
	subfieldofstudy-english = {Artificial Intelligence and Computer Science},
	fieldofstudy-czech = {Otevřená Informatika},
	subfieldofstudy-czech = {Základy umělé inteligence a počítačových věd},
	keywords-czech = {Finite-Horizon, MDP, Julia},
	keywords-english = {Finite-Horizon, MDP, Julia},
	day = 10,
	month = 1,
	year = 2021,
	specification-file = {ctutest-zadani.pdf},
% 	front-specification = true,
	front-list-of-figures = false,
	front-list-of-tables = false,
%	monochrome = true,
%	layout-short = true,
}

\ctuprocess

\addto\ctucaptionsczech{%
	\def\supervisorname{Vedoucí}%
	\def\subfieldofstudyname{Studijní program}%
}

\ctutemplateset{maketitle twocolumn default}{
	\begin{twocolumnfrontmatterpage}
		\ctutemplate{twocolumn.thanks}
		\ctutemplate{twocolumn.declaration}
		\ctutemplate{twocolumn.abstract.in.titlelanguage}
		\ctutemplate{twocolumn.abstract.in.secondlanguage}
% 		\ctutemplate{twocolumn.tableofcontents}
% 		\ctutemplate{twocolumn.listoffigures}
	\end{twocolumnfrontmatterpage}
}

% Theorem declarations, this is the reasonable default, anybody can do what they wish.
% If you prefer theorems in italics rather than slanted, use \theoremstyle{plainit}
\theoremstyle{plain}
\newtheorem{theorem}{Theorem}[chapter]
\newtheorem{corollary}[theorem]{Corollary}
\newtheorem{lemma}[theorem]{Lemma}
\newtheorem{proposition}[theorem]{Proposition}

\theoremstyle{definition}
\newtheorem{definition}[theorem]{Definition}
\newtheorem{example}[theorem]{Example}
\newtheorem{conjecture}[theorem]{Conjecture}

\theoremstyle{note}
\newtheorem*{remark*}{Remark}
\newtheorem{remark}[theorem]{Remark}

\setlength{\parskip}{5ex plus 0.2ex minus 0.2ex}

% Abstract in Czech
\begin{abstract-czech}
Do povědomí vědecké komunity se začíná dostávat nový programovací jazyk, Julia. Díky svému prudkému nárustu uživatelské báze a svému prostředí postavenému na open-source knihovnách, se dá očekávat, že se stane jedním z nejpopulárnějších jazyků. V této práci představíme Julii, MDP, teorii s nimi spjatou spolu s možnými přístupy k jejich řešení. Postupně ukážeme jejich výhody i nevýhody a nakonec se dostaneme k Finite-Horizon MDP, jako jedné z nejpokročilejších metod. Navrhneme design Finite-Horizon MDP rozhraní na základě koncepce z JuliaPOMDP/FiniteHorizonPOMDPs.jl \cite{FHPOMDP}, implementujeme jej a porovnáme s value iteration řešením DiscreteValueIteration.jl \cite{DVI} implementovaným v knihovně JuliaPOMDP's \cite{JuliaPOMDP}.
\end{abstract-czech}

% Abstract in English
\begin{abstract-english}
In the last couple of years, a new programming language, Julia, has emerged into the scientific community's consciousness. With the steep rise of its downloads and its environment built on open-source packages, it is expected to become one of the most popular languages. This project introduces Julia, MDPs, the theory behind it, and possible approaches to its solution. By reviewing its drawbacks and advantages, we gradually get to present Finite-Horizon MDPs as the advanced solution. We design Finite-Horizon MDP Interface according to declaration at JuliaPOMDP/FiniteHorizonPOMDPs.jl \cite{FHPOMDP}, implement the interface and benchmark it against JuliaPOMDP's \cite{JuliaPOMDP} value iterator solver DiscreteValueIteration.jl \cite{DVI}.

\end{abstract-english}

% Acknowledgements / Podekovani
\begin{thanks}
We thank our Supervisor Ing. Jan Mrkos for all the little things he has done for us in making this possible, particularly for his patience. Our thanks also belong to our family and our girlfriend for the support they provided us with, and finally, to FEE CTU for all joys and sorrows it enriched us with.
\end{thanks}

% Declaration / Prohlaseni
\begin{declaration}
I declare that this work is all my own work and I have cited all sources I have
used in the bibliography.

\medskip

Prague, \monthinlanguage{title} \ctufield{day}, \ctufield{year}

\vspace*{2cm}

Prohlašuji, že jsem předloženou práci vypracoval samostatně, a že jsem uvedl veškerou použitou literaturu.

\medskip

V Praze, \ctufield{day}.~\monthinlanguage{second}~\ctufield{year}
\end{declaration}

% Only for testing purposes
\listfiles
\usepackage[pagewise]{lineno}
\usepackage{lipsum,blindtext}
\usepackage{mathrsfs} % provides \mathscr used in the ridiculous examples

\begin{document}

\maketitle

%!TEX ROOT=ctutest.tex

\chapter{Introduction}

MDP's are one of the most well-known methods used for solving stochastic planning problems. In the last couple of years, a new programming language, Julia, has emerged into the scientific community's consciousness. While offering a high-level approach comparable to the one of Python, it also offers a similar speed of languages like C or C++ at the same time. As the new users come to this language, the demand for new tools rises. The subject of this work answers these calls and deals with implementing one of those methods, finite-horizon MDPs interface for Julia's POMDPs.jl package.

\section{Markov Decision Processes}
The Markov decision processes are a part of the optimization problem solvers.
The term MDP was probably first used in \cite{cite:1} and was derived from the name of Russian mathematician Andrey Markov who researched stochastic processes, on which are MDPs based \cite{wiki:1}.
The problems that are MDPs solving are described with the environment. The environment consists of the states and actions (with corresponding stochastic transitions) for which the agent either pays a cost or receives reward (negative cost). The agent's task is to minimize his reward. The problem can be finite or infinite according to the problem's description.

\section{Julia}
Julia is a young language that starts to receive traction for its features. According to https://star-history.t9t.io/#JuliaLang\julia the number of stars on its repository(https://github.com/JuliaLang/julia) tripled over the last 20 months and is rising steadily.
The development started in 2009 at MIT, and the project went public in 2012 [https://www.zdnet.com/article/is-julia-fastest-growing-new-programming-language-stats-chart-rapid-rise-in-2018/]. 
It combines the performance of low-level languages like Fortran or C with the high-level approach of Matlab, R and Python. In addition to that, Julia is dynamically typed, designed for parallelism and distributed computation, it offers multiple dispatch paradigms and is also reproducible while relying on community packages[https://julialang.org/].


\section{POMDPs.jl}
\textit{POMDPs.jl is a package providing a core interface for working with MDPs and POMDPs}\cite{JMLR:v18:16-300}x It offers the user a unified interface for custom problem definitions, problem interfaces for simpler ones, and multiple solvers for both MDPs and POMDPs. The package is managed by Zachary Sunberg and is under active development.


\section{Goal}
The goal of this project is to implement a finite horizon interface for MDPs in the POMDPs.jl package. To get to know how the package's environment works, develop possible test cases, solve them using a custom value iteration method, and then compare them with the already implemented method in POMDPs.jl. Devise a way to implement finite horizon for MDPs in the package and evaluate its correctness. Get feedback from the POMDPs.jl community, iterate the design, and submit a pull request to the corresponding repository.



%!TEX ROOT=ctutest.tex


\part{MDPs}

\chapter{Theory}

In the previous chapter, we have briefly introduced MDPs. In this chapter, we will build on that intuition by reviewing a small part of the theory behind them, and we will also show possible approaches to their solving.

\section{Environment}
The environment that we introduced so far contains states, actions, transition probabilities, and costs for each action. However, we did not assign any rules to it! 
Without these rules, one can not create any connection between states and actions or actions and transition probabilities.

The environmental rules could range anywhere from the ones applied in the crossword puzzle game (fully observable, deterministic, sequential, static, discrete with single-agent) up to the ones applied in the taxi driving planning (partially observable, stochastic, sequential, dynamic, continuous with multiple agents). We will not describe the meaning of all specific rules in this work, as it is out of its scope. In case the reader is interested, we refer to read  \cite{russel2010}.

\newpage


For the sake of simplicity, let us assume the following properties of the given environment:
\begin{description}
  \item[$\bullet$ Fully observable] The agent has all information needed about the environment.
  \item[$\bullet$ Stochastic] The movement of the agent is not certain. The result of movement can be different from the desired one as a result of noise,
  \item[$\bullet$ Sequential] The current action is going to have an impact on future actions,
  \item[$\bullet$ Static] The environment can not change.,
  \item[$\bullet$ Discrete] The environment has a finite number of states and actions.
\end{description}



\section{MDP and its features}
With environmental properties sorted out, let us move on to MDPs themselves. 
\\ 
\\
The definitions, algorithms and the construction of following chapters draw on from \cite{Kolobov2012}. 
  \\ \\

\begin{definition}[\textbf{Finite Discrete-Time Fully Observable Markov Decision Process (MDP)}]\label{def:MDP}
A fully observable MDP is a tuple $(S, A, D, T, R)$, where:
\begin{description}
  \item[$\bullet$ ] $S$ is the finite set of all possible states of the system, also called the state space;
  \item[$\bullet$ ] $A$ is the finite set of all actions an agent can take;
  \item[$\bullet$ ] $D$ is a finite or infinite sequence of the natural numbers of the form $(1, 2, 3, \ldots, T_{max})$ or $(1, 2, 3, \ldots)$ respectively, 
        denoting the decision epochs, also called time steps, at which agent needs to act;
  \item[$\bullet$ ] $T : S \times A \times S \times D \rightarrow [ \,0, 1] \,$ is a transition function, a mapping specifying the probability $T(s_1, a, s_2, t)$ of going to state $s_2$ if action $a$ is executed when the agent is in state $s_1$ at time step $t$;
  \item[$\bullet$ ] $R : S \times A \times S \times D \rightarrow R$ is a reward function that gives a finite numeric reward value $R(s_1, a, s_2, t)$ obtained when the system goes from state $s_1$ to state $s_2$ as a result of executing action $a$ at time step $t$.
\end{description}

\end{definition}

Every MDP is solving its problem according to its movement policies. As the number of policies itself can be high order polynomial, we are going to relax their definition as well.
\\ \\
\begin{definition}[\textbf{Markovian Policy}]
A probabilistic (deterministic) history-dependent policy $\pi: H \times A \rightarrow [ \,0, 1] \,(\pi: H \rightarrow A)$ is \textit{Markovian} if for any two histories $h_{s,t}$ and $h'_{s,t}$, both of which end in the same state $s$ at the same time step $t$, and for any action $a$, 
$\pi(h_{s,t}, a) = \pi(h'_{s,t}, a)$ $(\pi(h_{s,t}) = a$ if and only if $\pi(h_{s,t}) = a)$.
\end{definition}

Every state in every time step is then evaluated according to \textbf{value function}. A Markovian value function then corresponds to $ V: S \times D 
\rightarrow [ \,-\infty, \infty] \,$. In the following text we will refer to value function as $V(s, t)$ or $V(s)$.

The value function of a policy is then: \\ \\

\begin{definition}[\textbf{The Value Function of a Policy}]
Let $h_{s,t}$ be a history that terminates at state $s$ and time $t$. Let $R^{\pi_{h_{s,t}}}_{t'}$ be random variables for the amount of reward obtained in an MDP as a result of executing policy $\pi$ starting in state $s$ for all time steps $t'$ s.t. $t \leqslant t' \leqslant |D|$ if the MDP ended up in state $s$ at time $t$ via history $h_{s,t}$.The value function $V^{\pi}: H \rightarrow [ \,−\infty,\infty] \,$ \textit{of a history-dependent policy} $\pi$ is a utility function $u$ of the reward sequence $R^{\pi_{h_{s,t}}}_t, R^{\pi_{h_{s,t}}}_{t+1}, \ldots$ that one can accumulate by executing $\pi$ at time steps $t, t + 1,\ldots$  after history $h_{s,t}$. Mathematically, $V^{\pi} (h_{s,t}) = u(R^{\pi_{h_{s,t}}}_t , R^{\pi_{h_{s,t}}}_{t+1}, \ldots)$.
\end{definition}

Among the policies evaluated according to the previous rule, we will be able to find an \textbf{optimal MDP solution} denoted as $\pi^*$ with value $V^*$ called the optimal value function. Such optimal solution then satisfies $V^{*} (h) \leqslant V^{\pi} (h)$ for all histories $h \in H$.

One of the possible approaches to evaluate the value function is using Expected Linear Additive Utility.
\\
\begin{definition}[\textbf{Expected Linear Additive Utility}]
An \textit{expected linear additive utility} function is a function $u(R_t, R_{t+1}, \ldots) = E[ \,\sum_{t'=t}^{|D|} \gamma ^{t'−t} R_{t'}] \, = E[ \,\sum_{t'=0}^{|D|-t} \gamma^{t'} R_{t'+t}] \,$ that computes the utility of a
reward sequence as the expected sum of (possibly discounted) rewards in this sequence, where $\gamma \geqslant 0$
is the discount factor.
\end{definition}

This approach eliminates the possibility of multiple different solution values resulting from multiple stochastic runs of the same policy. It employs the expected value and does not need to perform vectors equality comparison as its results are scalars.

It turns out that this approach is even better as it guarantees a fundamental property of MDPs called \textbf{The Optimality Principle}, according to which \textit{among the policies that are evaluated by the expected linear additive utility, there exists a policy that is optimal at every time step.}

\section{MDPs techniques}

This part will briefly introduce a few fundamental algorithms for MDPs solving: Brute-Force algorithm, Policy Iteration, Value Iteration, and show a possible solution for its disadvantages: Prioritizations. \\
After reading this section, the reader should know the possibilities of solving MDP and be ready for the following section, where we will discuss the finite horizon's advantages.

\subsection{Brute-Force Algorithm}
As its name suggests, the Brute-force algorithm is the most naive algorithm, similar to all classes of problem-solving. In evaluating the problem, the method evaluates all possible combinations and chooses the best one.
Moreover, computer scientists are not using it in the vast majority of cases because the algorithm needs to evaluate $|S|^{|A|}$ policies.
$S$ in the previous equation represents states and $A$ represents actions.

However, as the number of states, or actions, rises, such evaluation becomes enormous. Another problem comes from the fact that actions in the environment are not deterministic and often cyclic. Both of these problems results in a steep complexity rise.

That is why another approach comes in.

\subsection{An iterative approach to policy evaluation}

The evaluation of cyclic environments requires a suitable equation that captures all movements, transitions, and rewards.
That means that every state's value function should correspond to the sum of rewards for moving towards successor states. Moreover, the agent should maximize its reward by getting to the goal, so the successor state's value function multiplied by its probability should also appear in the equation. Let us state the formula as following:\\
\begin{equation}
\begin{aligned}
V^{\pi} (s) & = 0 && \text{(if $s \in G$)} \\
& = sum_{s' \in S} T(s, \pi (s), s') [ \,C(s, \pi(s), s') + V^{\pi} (s')] \, && \text{(otherwise)}
\end{aligned}
\end{equation}

This system of linear equations reduced the complexity of the previous approach. It \textit{can} be solved using Gaussian Elimination in $O(|S|^3)$. It is still not efficient enough, but we use its idea to solve it with \textbf{an iterative approach}.

The iterative approach does precisely what it says. 
It starts with rough estimations and continuously works its way up to the same solution as the linear system. Besides, this solution is optimal \cite{Kolobov2012}. As it stores the previous iteration, it is a part of dynamic programming. Every iteration of this algorithm runs in $O(|S|^2)$.

\LinesNumbered
\begin{algorithm}[H]
\SetAlgoLined
//Assumption: $\pi$ is proper (ends up in a goal state) \\
initialize $V^{\pi}_0$ arbitrarily for each state \\
$n \xleftarrow{} 0$ \\
\Repeat{$ max_{s\in S} residual_n (s) \leq \epsilon$}{
$n \xleftarrow{} n + 1$ \\
\ForEach{$ s \in S $}{
compute $V^{\pi}_n (s) \xleftarrow{} \sum_{s' \in S} T (s, \pi (s), s') [ \,C(s, \pi (s), s') + V^{\pi}_{n−1} (s') ] \,$ \\
compute $residual_n (s) \xleftarrow{} |V^{\pi}_n (s) − V^{\pi}_{n−1} (s)| $ \\
}
}
return $V^{\pi}_n$
\caption{Iterative Policy Evaluation}
\end{algorithm}

We know how to evaluate the environment's states given some policy, but we did not yet describe how to choose the best policy for a given evaluation. 

\subsection{Policy iteration}
Given that the result of iterative policy evaluation is optimal after an undefined number of iterations for a specific policy, we can further improve it by iterating policy evaluation and policy improvement.
For every iteration, we have \textit{almost} optimal policy evaluation for the previous policy. We can improve the policy by iterating over possible actions from all states and choosing that action if it has a lower cost than the previous one. The algorithm stops when the policy remains without a change at the end of the following iteration.

\LinesNumbered
\begin{algorithm}[H]
\SetAlgoLined
initialize $\pi_0$ to be an arbitrary proper (ending in the goal state) policy \\
$n \xleftarrow{} 0$ \\
\Repeat{$\pi_n == \pi_{n−1}$ }{
    $n \xleftarrow{} n + 1$ \\
    Policy Evaluation: compute $V^{\pi_{n−1}}$ \\
    Policy Improvement: \\
    \ForEach{state $s \in S$}{
        $ \pi_n (s) \xleftarrow{} \pi_{n−1} (s) $ \\
        $ \forall a \in A$ compute $Q^{(V^{\pi_{n−1}})} (s, a) $ \\
        $ V_n (s) \xleftarrow{} min_{a \in A} Q^{(V^{\pi_{n−1}})} (s, a) $ \\
        \uIf{$ Q^{(V^{\pi_{n−1}})} (s, \pi_{n−1} (s)) > V_n (s)$}{
            $ \pi_n (s) \xleftarrow{} argmin_{a \in A} Q^{(V^{\pi_{n−1}})} (s, a)$
        }
    \textbf{end}
    }
}
return $\pi_n$
\caption{Policy Iteration}
\end{algorithm}

Until now, whenever we talked about policy iteration, we mentioned that the initial policy has to be proper.
That is because if we do not meet the initial condition, the policy evaluation step will diverge.
Nevertheless, another algorithm can solve this drawback.


\subsection{Value iteration}
Value iteration focuses on improving value functions of states, instead of evaluating the policy. The algorithm creates policy at its very end as the action that minimizes the cost in given state. Thanks to this property, the initial policies are not used in the algorithm, and can not lead to divergence.

\textit{The algorithm is based on \textbf{Bellman equations}, which mathematically express the optimal solution of an MDP.}

\begin{equation}
\begin{aligned}
V^* (s) & = 0 && \text{(if $s \in Q$)} \\
& = min_{a \in A} Q^* (s, a) &&  (s \notin G) \\ 
Q^* (s, a) & = \sum_{s' \in S} T(s, a, s')[ \,C(s, a, s') + V^*(s')] \, \\
\end{aligned}
\end{equation}

The equation's interpretation is minimizing the optimal value function of a given state. It can have value 0 if the state is among goal states, or value minimizing the result of performing given action and then following optimal policy if the state is not.

In order to approximate $V^* (s)$, the algorithm uses the so-called \textbf{Bellman backup} to improve the current $V_n (s)$ with $V_{n-1} (s')$.


\begin{equation}
\begin{aligned}
V_n (s) \xleftarrow{} min_{a \in A} \sum_{s' \in S} T(s, a, s') [ \,C(s, a, s') + V_{n - 1} (s')] \,
\end{aligned}
\end{equation}

Value iteration is iteratively improving its value function estimations and is guaranteed to converge to the optimal solution \cite{Kolobov2012}. The stopping criterion can be specified as either max residual, maximal difference between consecutive state value function estimations, or as the number of iterations.

\LinesNumbered
\begin{algorithm}
\SetAlgoLined
initialize $V_0$ arbitrarily for each state \\
$n \xleftarrow{} 0$ \\ 
\Repeat{ $max_{s \in S} \: residual_n $ (s) < $ \epsilon $ }{
    $n \xleftarrow{} n + 1$ \\
    \ForEach{$ s \in S $ }{
        compute $V_n (s)$ using Bellman backup at $s$ \\ 
        compute $residual_n (s) = |V_n (s) - V_{n-1} (s)|$
        }
    }
return greedy policy: $\pi^{V_n} (s) = argmin_{a \in A} \sum_{s' \in S} \tau(s, a, s')[ \,C(s, a, s') + V_n (s')] \,$
\caption{Value Iteration}
\end{algorithm}


\subsection{Prioritization}
\textit{One of the most significant drawbacks of Value Iteration is that it requires full sweeps of the state space}. This drawback results in many unnecessary value function evaluations because its value remains the same as the value function updates are heading from goal states at first. It can take lots of iterations to evaluate all the successor state's value functions. Furthermore, due to cyclic moves, the speed of convergence among states can differ.

Both of these problems often result in flawed performing algorithm executions.

Logical improvement is to, instead of iterating whole state spaces, iterate over each state separately. This has two consequences:
\begin{enumerate}
    \item We have to develop or choose an algorithm that prioritizes states and 
    \item does not starve some of them (meaning every state gets updated accordingly).
\end{enumerate}

This also means that we can no longer use the number of iterations as a termination condition because we do not know which states' value function and how often, were updated. The only terminal condition remains maximum residual.

\cite{Kolobov2012} formalizes the intuition as follows:

\LinesNumbered
\begin{algorithm}
\SetAlgoLined
initialize $V$ \\
initialize priority queue $q$ \\
\Repeat{ termination }{
    select state $s' = q.pop()$ \\
    compute V(s') using a Bellman backup at s'\\
    \ForEach{predecessor s of s', i.e. $\{s|\exists a [\, T(s, a, s') > 0] \,\}$}{
        compute $priority(s)$\\
        $q.push(s, priority(s))$\\
        }
    }
return greedy policy $\pi^{V}$
\caption{Prioritized Value Iteration}
\end{algorithm}

We are going to introduce a few priority metrics whose features generally differ and may be even diverging under specific conditions. For details, head over to \cite{Kolobov2012}

\subsubsection{Prioritized Sweeping} 

\textit{Prioritized sweeping estimates the expected change in the value of a state if a backup were to be performed on it now.}

\begin{equation}priority_{PS} (s) \xleftarrow{} max \Big\{ priority_{PS} (s), max_{a \in A} \big\{ T(s, a, s') Res^{V}(s')\big\} \Big\} \end{equation}

\subsubsection{Improved Prioritized Sweeping}

Improved Prioritized Sweeping employs the idea that states with fast converging value functions are to be prioritized. The consequence is that the algorithm first iterates the states near the goal and moves to other states only after the change between state iterations becomes small.

\begin{equation}priority_{IPS} (s) \xleftarrow{} \frac{Res^{V} (s)}{V (s)} \end{equation}

\subsubsection{Focused Dynamic Programming}

Focused Dynamic Programming is a particular case of prioritization techniques used when the start state $s_0$ is known. In such cases, the start case's knowledge can be employed and added as a penalty factor.
\newpage
\begin{equation}priority_{FDP} \xleftarrow{} h(s) + V(s)\end{equation}


where:
\begin{description}[1cm]
  \item[h(s)] \textit{is lower bound on the expected cost of reaching s from $s_0$}, and
  \item[V(s)] is regular value function for given state s.
\end{description}

In the previous sections, we have very briefly introduced the notion of MDPs and their solving methods. First, we showed the Brute-Force algorithm, which naively evaluated all possible policies. Then we discussed the value and policy iterations that improved the performance but still did lots of useless evaluations. And finally, we presented a solution that intelligently iterates through prioritized states. 

\section{Finite-Horizon MDP}

With all the necessary theory layed out. Let us present the Finite-Horizon MDP.\\ \\

\begin{definition}[\textbf{Finite-Horizon MDP}]
A finite-horizon MDP is an MDP as described in ~\ref{def:MDP} with a finite number of time steps, i.e., with $|D| = T_{max} < \infty$.
\end{definition}


The Infinite-Horizon MDPs discussed so far are a slightly different class of problems from Finite-Horizon MDPs. However, the solution of Finite-Horizon is somewhat similar to the prioritized value iteration of Infinite-Horizon MDPs. 
Moreover, each Infinity-Horizon MDP can be transformed into Finite-Horizon one.

This is critical because the Finite-Horizon MDP has an acyclic state space, and \textbf{the acyclic MDPs can be solved optimally using only one backup if used optimal backup order.} \cite{Kolobov2012}

On the other hand, the transformation to Finite-Horizon MDP polynomially increases the state space.

As the finite horizon number is known, we can evaluate all states' value functions from maximum horizon time and work our way down to starting time. This way, we evaluate each state's successor's value functions, employing optimal backup order and finding the optimal policy on the first pass.

In conclusion, in cases where the Infinite-Horizon methods are not sufficient, the MDPs can be transformed into Finite-Horizon ones. However, while solving the MDP in one pass, this transformation also blows up the state space.



\chapter{Implementation}


tady by se asi hodilo to rozdelit na jednotlive casti

\section{FiniteHorizonPOMDPs.jl}

jeste nevim kam presne zaradit interface, protoze to je pro MDP i pro POMDP

\section{FiniteHorizonDiscreteValueIteration.jl}






This chapter presents the solution design, implementation, and shows its benchmark.
As a goal for this project, we set to implement Finite-Horizon Solver for MDPs only.

\section{Solution Design}
The package's design is created in line with proposed declarations present in the original repository's README.md and discussed with Zachary Sunberg.

The goal of the project is to create a POMDPs.jl-compatible interface for defining MDPs and POMDPs with finite horizons, and in particular:
\begin{itemize}
    \item Provide a way for value-iteration-based algorithms to start at the final-stage and work backward
    \item Be compatible with generic POMDPs.jl solvers and simulators
    \item Provide a Finite-Horizon wrapper for Infinite-Horizon MDPs.
    \item Be compatible with other interface extensions like constrained POMDPs and mixed observability problems.
\end{itemize}

\subsection{Interface}
Moreover, the interface should contain or require the implementation of the following methods:
\begin{itemize}
    \item \textit{HorizonLength(::Type{<:Union{MDP,POMDP}}) = InfiniteHorizon()}
    \begin{itemize}
        \item \textit{FiniteHorizon}
        \item \textit{InfiniteHorizon}
    \end{itemize}
    \item \textit{horizon(m::Union{MDP,POMDP})::Int}
    \item \textit{stage\_states(m::Union{MDP,POMDP}, t::Int)}
    \item \textit{stage\_stateindex(m::Union{MDP,POMDP}, t::Int, s)}
    \item \textit{stage\_actions(m::Union{MDP,POMDP}, t::Int, [s])}
\end{itemize}

\subsection{Utilities}
And also the following utility:

\begin{itemize}
    \item \textit{fixhorizon(m::Union{MDP,POMDP}, T::Int) creates one of}
    \begin{itemize}
        \item \textit{FiniteHorizonMDP{S, A} <: MDP{Tuple{S,Int}, A}}
        \item \textit{FiniteHorizonPOMDP{S, A, O} <: POMDP{Tuple{S,Int}, A, O}}
    \end{itemize}
\end{itemize}

\section{Implementation}

The implementation is at \cite{FHPOMDP} or APENDIX.

The Finite Horizon algorithm evaluates a given MDP problem using Value Iteration.  

\subsection{Solution approach}

 The Current solution consists of a solver \textit{mysolve(mdp)} iterating and evaluating epochs and \textit{FiniteHorizonPolicy} structure storing its results (value function matrix, matrix of Qs, policy, map of actions, MDP instance). 
 
%  \textbf{Finite-Horizon Policy}
% \begin{description}[1cm]
%     \item $Q$ 
%     \item $V$  
%     \item $policy$ 
%     \item $actions$ 
%     \item $include\:Q$ 
%     \item $MDP$ 
% \end{description}
 
 \textit{Mysolve} takes the Finite-Horizon MDP as input, initializes policy structure and value function matrix, and iterates from \textit{$(maximum\_horizon-1)^{th}$} to \textit{$1^{st}$} epoch. Each epoch is then evaluated using a value iteration algorithm, which is initialized with the following epoch's value function matrix. Results of each epoch are stored in \textit{FiniteHorizonPolicy}.
 
 \LinesNumbered
\begin{algorithm}
\SetAlgoLined
initialize $FiniteHorizonPolicy$ \\
initialize $V$ \\
\For{$epoch = horizon - 1;\ epoch > 0;\ epoch = epoch - 1$}{
    set \textit{epoch} global \\
    run 1 iteration of \textit{value\_iteration(MDP, epoch, V)} \\
    update \textit{FiniteHorizonPolicy} \\
}
return \textit{FiniteHorizonPolicy}
\caption{Finite-Horizon MDP Solver mysolve}
\end{algorithm}

In the current version of the algorithm, we have to set the epoch global to pass it to interface functions. The next planned version will no longer contain the global variable.

\subsection{Defining MDP}

To define the MDP instance, we have to define the methods declared in \textit{JuliaPOMDP/POMDPs.jl} \cite{JMLR:v18:16-300} or from another library's package. The Finite-Horizon MDP Solver further requires: \textit{stage\_states}, which returns states of a given epoch, \textit{stage\_actions}, which return actions of a given epoch, and \textit{stage\_stateindex}, which return indices of a given state for a given epoch. For example of properly defined MDP see directory \textit{/test/instances}.
 

\subsection{Future plan}

The following releases' goals are to improve the performance, implement Infinite-Horizon MDP wrapper, and implement Finite-Horizon solver for POMDPs.


\section{Benchmark}
To prove the efficiency of implemented Finite-Horizon solver, we present a benchmark of Finite-Horizon MDP instance solved by
\begin{itemize}
    \item Value Iteration implemented in JuliaPOMDP/DiscreteValueIteration.jl
    \item Finite-Horizon Value Iteration solver implemented in this project in JuliaPOMDP/FiniteHorizonPOMDPs.jl package.
\end{itemize}

\subsection{MDP Instances}

To compare the results obtained, the instance has \textit{num\_of\_states * finite\_horizon} states for both solvers.

The instances were defined as 1D GridWorld problem with goal states on its right and left ends. The possible actions are \textit{move to left} and \textit{move to right}. The cost of such a move is 1, and the reward for moving to the goal is -10. Goals are terminal states and can not be left. The discount factor is one and the noise 0.6.

The benchmark were be performed on 3 different instance sizes:
\begin{itemize}
    \item \textbf{Instance 1} $states = 999, horizon = 500$,
    \item \textbf{Instance 2} $states = 99, horizon = 50$, and
    \item \textbf{Instance 3} $states = 9, horizon = 5$.
\end{itemize}

\subsection{Comparison}

The benchmark resulted in following:

\subsubsection{Instance 1}

\begin{itemize}
    \item \textbf{Value Iteration} 183.729947 seconds (3.48 G allocations: 244.793 GiB),
    \item \textbf{Finite-Horizon Value Iteration}   0.566122 seconds (7.17 M allocations: 579.764 MiB)
\end{itemize}

\subsubsection{Instance 2}

\begin{itemize}
    \item \textbf{Value Iteration} 1.082094 seconds (4.41 M allocations: 292.613 MiB)
    \item \textbf{Finite-Horizon Value Iteration}   0.199167 seconds (241.17 k allocations: 14.231 MiB)
\end{itemize}

\subsubsection{Instance 3}

\begin{itemize}
    \item \textbf{Value Iteration} 0.643639 seconds (1.08 M allocations: 51.450 MiB)
    \item \textbf{Finite-Horizon Value Iteration}   0.200691 seconds (172.53 k allocations: 8.653 MiB)
\end{itemize}  

We tested each benchmark's results to be the same.

The benchmarks show that the acyclic graphs with the optimal backup policy result in optimal policy and perform precisely both time-wise and memory-wise. We expect the results to be even better in the following versions.

\newcommand{\norm}[1]{\left\lVert#1\right\rVert}


\chapter{Partially Observable Markov Decision Processes}


Up until now, we have dealt with fully observable MDPs. In MDPs, the agent has the perfect knowledge about its position. In practice, however, fully observable environments are scarce. Real-world agents often have only a limited domain of knowledge about their surroundings, making their state uncertain. These environment models are called \textbf{partially observable MDPs} (or POMDPs). With imperfect state information, the functions defined in MDPs are no longer valid, and we have to define new, more powerful methods dealing with stochastic probability distributions rather than with deterministic ones.

 In the following chapter, we will mainly draw on from \cite{russel2010}, but also from \cite{Shani2013}, \cite{pbvi} and \cite{Walraven19}.

\section{POMDP Models}

The POMDPs draw on from MDPs, and thus, some of the model methods remain the same. POMDPs, similar to MDPs, have the transition model $P(s'|s,a)$, states $S$, actions $A(s)$ and reward model $R(s)$ defined. Besides that, the POMDPs also define a new model, which describes the possibility of receiving observations of the agent's surroundings, called the observation function $P(o|s', a)$ and set of observations possible $\Omega$. Unlike MDPs, the POMDPs define the starting position $b_0$ as the initial distribution instead of a deterministic state. 


\begin{figure}[h]
\begin{definition}[\textbf{Partially Observable Markov Decision Process (POMDP)}]\label{def:POMDP}
POMDPs \cite{Shani2013} are formally defined as a tuple $\{S, A, T, R, \Omega, O, b_0\}$, where:
\begin{description}
  \item[$\bullet$ ] \textit{S, A, T, R} are the same methods as for fully observable MDPs, often called the \textit{underlying} MDP of the POMDP.
  \item[$\bullet$ ] $\Omega$ is a set of possible observations. For example, in the robot navigation problem, $\Omega$ may consist of all possible immediate wall configurations in a room.
  \item[$\bullet$ ] \textit{O} is an observation function, where $O(a, s', o) = P(o|s', a)$ is the probability of observing \textit{o} given that the agent has executed action \textit{a}, reaching state \textit{$s'$}. \textit{O} can model
robotic sensor noise, or the stochastic appearance of symptoms given a disease.
  \item[$\bullet$ ] $b_0$ is an initial state distribution.
\end{description}

\end{definition}
\end{figure}

% \section{Belief state and its updates}

The initial state distribution $b_0$ and the other state distributions mentioned above are called belief states. The belief states are $|S|$ long vectors describing the probability distribution over all states in POMDPs. The probability of a single state is written as $b(s)$. The update of belief is calculated for each action and observation as the conditional distribution given the sequence of actions executed and observations received so far starting from the initial belief state. We formulate the update of each state as:

\begin{equation} \label{eq:bao} b_o^a(s') = \dfrac{P(o|a, s')}{P(o|b, a)} \sum_{s \in S} P(s'|s, a) b(s),\end{equation}

where $P(o|b, a)$ corresponds to the probability to observe observation \textit{o} after executing action \textit{a} in belief \textit{b}. This probability is calculated as follows:

\begin{equation} P(o|b, a) = \sum_{s' \in S} P(o|a, s')\sum_{s \in S} P(s'|s, a)b(s).\end{equation}

This approach to calculating the normalizing constant is computationally more demanding than other approaches. However, it allows skipping the computation of the inner sum in cases where the probability is zero.

If deemed appropriate, we can replace the normalizing constant with another normalizing constant $c$. In such cases, we obtain the impossibility of receiving an observation $o$ after executing action $a$ at the very end of the belief update.


\section{Value functions for POMDPs}
In MDPs, the value function computes the value for each discrete state. With POMDPs having a continuous space of belief vectors, we need to alter the model method further. Instead of computing a policy for a single state, we need to compute utility vectors of length $|S|$, called \textbf{$\alpha$-vectors}, for belief states. The value of belief state then becomes $\sum_{s \in S} b(s) \alpha_p(s)$, where $\alpha_p(s)$ is the utility of executing a policy p in state s. This linear combination can be interpreted as a hyperplane over the belief space \cite{pbvi}. The optimal policy in a given belief then becomes an action a, whose $\alpha$-vector maximizes the dot product with a given belief:

\begin{equation} V(b) = V^{p^*}(b) = \operatorname*{max}_p b \cdot \alpha_p \end{equation}

More importantly, the previous statements mean that the set of $\alpha$-vectors create a piecewise linear and convex hyperplane \cite{russel2010}.

In general, the value function can be formulated as an iterative exact value iteration algorithm \cite{Shani2013}:

\begin{equation} V(b) = \max_{a \in A} [R(b, a) + \gamma \sum_{b' \in B} T(b, a, b') V(b')].\end{equation}

Or, in terms of vector operations and operations on sets of vectors \cite{Shani2013}, as:

\begin{equation} \label{eq:1} V' = \bigcup_{a \in A} V^a \end{equation}
\begin{equation} \label{eq:2} V^a = \bigoplus_{o \in \Omega} V^{a,o} \end{equation}
\begin{equation} \label{eq:3} V^{a,o} = \left\{\dfrac{1}{|\Omega|} r_a + \alpha^{a, o} : \alpha \in V \right\} \end{equation} 
\begin{equation} \label{eq:4} \alpha^{a, o}(s) = \sum_{s' \in S} O(a, s', o) T(s, a, s') \alpha(s'),\end{equation}

In each iteration of the exact value iteration algorithm, the value function is updated across the entire belief space. For each possible action, observation and $\alpha$-vector of old value function, the Eq. \ref{eq:4} computes new $\alpha$-vector, costing $O(|V| \times |A| \times |\Omega| \times |S^2|)$. The alpha vector is then summed with the reward corresponding to a given action in Eq. \ref{eq:3}, cross-summed across the observation space in \ref{eq:2} and unioned across the action space in \ref{eq:1}, adding another $O(|A| \times |S| \times |V|^{|\Omega|}$ to the resulting complexity of a single iteration which is $O(|V| \times |A| \times |\Omega| \times |S^2|) + |A| \times |S| \times |V|^{|\Omega|}$ 

 Keeping in mind the overwhelming size of POMDPs' belief space, the algorithm's performance quickly vanishes with the growing size of the problems.


\section{Point-Based Value Iteration}

Most of the POMDP problems are unlikely to reach most of the points in the belief space. To reduce the complexity of solving POMDPs, approximation algorithms are at hand. In the past, researchers introduced several approximation algorithms (\cite{10.2307/171496} suggests creating a grid-based belief set, \cite{Hauskrecht00} suggests creating a set of reachable beliefs). These algorithms, however, rely on naive approximations and underperform in the result. For example, creating grid-based belief sets turned out to be inaccurate in sparse grids or computationally expensive in the case of dense grids.

Adopting yet another strategy can overcome the problems described above. Given the most probable belief states, we can focus the solver on finding only their corresponding $\alpha$-vectors and thus reduce the computation overhaul. This algorithm is called Point-Based Value Iteration (PBVI).

The PBVI \cite{pbvi} is an anytime algorithm - it starts from an initial belief $b_0$ and iteratively improves its value function and expands its belief space. The belief space $B = \{b_0, b_1, \ldots, b_m\}$ stores the the most probable belief vectors. The value function updates each belief vector, but unlike other approximation algorithms, such as grid-based VI, its value function spans over the whole belief space instead of only one belief vector.

\LinesNumbered
\begin{algorithm}[H]
\SetAlgoLined
$B \xleftarrow{} {b_0}$\\
\While{\textit{V has not converged to $V^*$}}{
    $Improve(V, B)$\\
    $B \xleftarrow{} Expand(B)$\\
}
\caption{PBVI}
\end{algorithm}

Thanks to being an anytime algorithm, the solver can be stopped whenever needed, effectively exchanging computation time and solution quality. It is also possible to choose how big the maximum gap between iterations of value updates will be.


\subsection{Improve function}

The value function improvement, or backup, is an adjustment of the value update from Eqs. \ref{eq:1} - \ref{eq:4} It maintains only one $\alpha$-vector per belief vector, It also chooses only the best $\alpha$, pruning the dominated vectors twice, at each \textit{argmax} expression, and reducing the complexity of algorithm. The backup can be compactly written as:

\begin{equation} \label{eq:5} backup(V, b) = \operatorname*{argmax}_{\alpha_{a}^{b}:a \in A, \alpha \in V} b \cdot \alpha_{a}^{b}\end{equation}
\begin{equation} \label{eq:6} \alpha_{a}^{b} = r_a + \gamma \sum_{o \in \Omega} \operatorname*{argmax}_{\alpha^{a, o}:\alpha \in V} b \cdot \alpha^{a, o},\end{equation}

where

\begin{equation} \label{eq:7} \alpha^{a, o} (s) = \sum_{s' \in S} \alpha (s') P(o|a, s') P(s'| s, a).\end{equation}

\LinesNumbered
\begin{algorithm}[H]
\SetAlgoLined
\Repeat{\textit{V has converged} }{
    \ForEach{$b \in B$}{
        $\alpha \xleftarrow{} backup(b, V)$ \tcc{execute a backup operation on all points in B in arbitrary order}
        $V \xleftarrow{} V \bigcup \{\alpha\}$
    }
}
\tcc{repeat the above until V stops improving for all points in B}
\caption{PBVI Improve}
\end{algorithm}


This form starts with the same $\alpha$-vector update Eq. \ref{eq:7} as in the exact value update Eq. \ref{eq:4} with the complexity of $O(|S|^2 \times |A| \times |V| \times |\Omega|)$. The summation and dot product in Eq. \ref{eq:5} then takes further $O(|\Omega| \times |S|)$ and $O(|S)|)$ for adding a reward vector. With another $O(|S|)$ operations for the dot product in \ref{eq:5} the complexity of full point-based backup requires $O(|S|^2 \times |A| \times |V| \times |\Omega|) + |A| \times |S| \times |\Omega|)$.

The PBVI algorithm, unlikely the original value update, does not need to cross-sum the $\alpha$-vectors. Furthermore, the $\alpha^{a, o}$-vectors can be cached, because they are independent of the current belief b. Thus, computing the backup of whole $|B|$, with $\alpha^{a, o}$ cached requires only $O(|A| \times |\Omega| \times |V| \times |S|^2 +
|B|\times|A|\times|S|\times|\Omega|)$, instead of $O(|V| \times |A| \times |\Omega| \times |S^2| + |A| \times |S| \times |V|^{|\Omega|})$ in case of exact backup (Eqs. \ref{eq:1} - \ref{eq:4}).


The backup runs until the convergence of pair $\alpha$-vectors or until a predefined number of iterations.

\subsection{Expand function}

After the completion of function improvement, the algorithm executes the belief space expansion. At this point, the goal is to reduce the error bound as much as possible. The error-bound reduction is performed by greedily expanding the belief set with a new furthest belief accessible from each stored belief. The distance function is defined as follows, with $L$ being the chosen distance metric:

\begin{equation}|b' - B|_L = \operatorname*{min}_{b \in B} |b - b'|_L,\end{equation}

and the expanded belief results from :

\begin{equation}b' = \operatorname*{max}_{a, o} |b^{a, o} - B|_L\end{equation}

The choice of the distance metric is not crucial as the results appear to be identical, authors of the algorithm recommend to add 1 new belief per 1 old \cite{pbvi}.


\LinesNumbered
\begin{algorithm}[H]
\SetAlgoLined
$B_{new} \xleftarrow{} B$ 
\ForEach{$b \in B$}{
    $Successors(b) \xleftarrow{} \{b^{a, o}|Pr(o|b, a) > 0\}$\\
    $B_{new} \xleftarrow{} B_{new} \bigcup \operatorname*{argmax}_{b' \in Successors(b)} \norm{B,b'}_L$ \tcc{add the furthest successor of b}
}
\caption{PBVI Expand}
\end{algorithm}



\section{Alternative Point-Based algorithms} \label{PBVI}

The generic PBVI, while being simple and somewhat clever with its belief set expansion, becomes cumbersome when solving large POMDP instances. However, we often have to expand to large belief sets. Furthermore, we want to keep the possibility to compute a compact value function. To cope with these circumstances, the researchers developed various alternatives and heuristics.


\subsection{Perseus}

The idea of Perseus \cite{perseus} comes from the Achilles heel of PBVI. The PBVI is cleverly expanding its belief space with the most probable belief states. However, such an approach is significantly demanding. On the other hand, the Perseus algorithm shows that even running random trials and exploring a large number of belief vectors may be more efficient if used with clever value updates.


\subsection{HSVI}
The Perseus algorithm randomization is well suited for small and mid-sized problems but fails in larger ones. The intractability is caused by storing an enormous number of belief vectors. The HSVI \cite{hsvi} algorithm solves this issue by employing a straightforward yet effective heuristic that helps cut down the gap between the upper and lower bounds on the optimal value function. The HSVI stores the belief vector visits and backs them up in the reversed order.


\subsection{FSVI}
The FSVI \cite{fsvi} uses another heuristic creating new trajectories by using the best action in the underlying MDP. Such heuristics focus on searching high reward policies but fail to find how information retrieval is essential.


\subsection{SARSOP}
SARSOP \cite{sarsop}, unlike the others, focuses on calculating a cover of optimal belief space, focusing the search into a smaller area, and removing areas, which the optimal policy will not visit.


% \begin{algorithm}[H]
% \SetKwFunction{FMain}{Main}
% \SetKwProg{Fn}{Function}{:}{}
% \Fn{\FMain{$f$, $a$, $b$, $\varepsilon$}}{
% \LinesNumbered
% \SetAlgoLined
%     $B_{new} \xleftarrow{} B$ 
%     \ForEach{$b \in B$}{
%         $Successors(b) \xleftarrow{} \{b^{a, o}|Pr(o|b, a) > 0\}$\\
%         $B_{new} \xleftarrow{} B_{new} \bigcup \operatorname*{argmax}_{b' \in Successors(b)} \norm{B,b'}_L$ \tcc{add the furthest successor of b}
%     }
% }
% \caption{PBVI Expand}
% \end{algorithm}



\chapter{Implementation}


\section{solvers}

\subsection{approximate solvers}

\subsection{prunning}

jeste si nejsem jistej jestli ty dve predchozi kapitoly dam do implementace nebo do teorie

\section{PointBasedValueIteration.jl}
popsani toho jak to fungovalo predtim a jak jsem to upravil

\section{FiniteHorizonPointBasedValueIteration.jl}
popis implementace







Implementation

PBVI

The PBVI algorithm is stored in repo PointBasedValueIteration.jl and is partially based on former algorithm stored in the same place. The former algorithm, however, worker only for 2 states and completely omited the expansion phase, as the initial belief space was initiated with discrete distribution of beliefs.


The algorithm accepts problems defined in JuliaPOMDP/POMDPs.jl interface.



To some extent, the algorithm was written with list comprehensions or vectorizations, where it made sense. The utilization of vectorization was limited due to Julia's speed, often making the vectorizations slower than using for loops.



The PBVI's setting are handed over by the PBVISolver structure, which accepts following parameters:
num\_iteration, precision, verbose

The solve parameters are in line with POMDPs interface. That is, method solve accepts two parameters, solver's settings and POMDP to be solved. 


FIVI - TO BE COMPLETED 



















\input{ctutest-EX-problems}

\input{ctutest-EX-benchmark}

\chapter{Validation}

popis validaci a testu

porovnani s ostatnimi solvery, pouziti simulaci





\chapter{Optimizing and Profiling}




OPTIMIZATIONS

The optimizations were done using Profiler.jl which records the bottlenecks in the algorithm.

It turns out that even if the algorithm itself is optimized, the biggest bottleneck are user-defined functions, if not optimized correctly. Furthermore, the user-defined POMDPs definition bottlenecks usually have the same bottleneck, that is - they return values in list. Such functions are allocation small memory blocks each time they are called, up to million times, resulting in massive overhead in memory consumption. Such allocation can be however easily fixed by replacing them with generator expressions. In Julia, this is achieved by replacing square brackets with round brackets.


Optimizing allocations from pushing to preallocations.
Optimizing allocations by precomputing array only once instead of each time in for loop.


\chapter{Conclusion}

This thesis deals with the contribution to open-source Julia package \newline \textit{JuliaPOMDP/FiniteHorizonPOMDPs.jl} \cite{FHPOMDP} as a part of the more splendid library \textit{JuliaPOMDP} \cite{JuliaPOMDP}. Its task is to implement an interface for Finite-Horizon evaluations of MDPs. We introduced MDPs and their features, then built the essential MDP solvers theory, showed their advantages and disadvantages, and further improved these methods with prioritization. On its intuition, we finally presented the Finite-Horizon MDPs and summarized why it could be beneficial to use them instead of the Infinite-Horizon ones. 

We also benchmarked our package against the value iteration solver and showed better time and memory performance.


We believe that the package will be of fair use to everyone out there searching for a way to solve their Finite-Horizon MDP  problem with an already written package that is also efficient. The \textit{JuliaPOMDP} \cite{JuliaPOMDP} is an excellent designed general package for work with MDPs, and with the advent of Julia, it will get even more attention.

In future work, our main goal is to add POMDPs solver, improve the algorithm's efficiency and to create a package rewarding the user for its use.

In conclusion, our project is on its way to deliver a great package and further improve its library environment. With the rise of Julia, it is the best time to contribute to Julian open-source packages.

\appendix

% \printindex


\bibliographystyle{amsalpha}
\bibliography{ctutest}

% \ctutemplate{specification.as.chapter}

\end{document}